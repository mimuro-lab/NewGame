\href{https://qiita.com/wakaba130/items/faa6671bd5c954cb2d02\#\%E4\%BD\%BF\%E3\%81\%84\%E6\%96\%B9doxygen\%E5\%AE\%9F\%E8\%A1\%8C\%E7\%B7\%A8}{\texttt{ 最近覚えた便利アプリ\mbox{[}doxygen\mbox{]}}}を参考にしている。

1、\+Step1の出力するドキュメントの置き場所を設定する。

2、\+Wizeard/\+Projectウィンドウでそのドキュメントの名前を入力し、ドキュメントを生成したい対象のソースコードがあるディレクトリを参照する。

3、\+Export/\+Projectウィンドウで\+O\+U\+T\+P\+U\+T\+\_\+\+L\+A\+N\+G\+U\+A\+G\+Eを\+Japaneseにし、\+Export/\+Inputウィンドウで\+I\+N\+P\+U\+T\+\_\+\+E\+N\+C\+O\+D\+I\+N\+Gを\+C\+P932に変更する。

4、\+Runウィンドウの\+Run doxygenをクリックする。 